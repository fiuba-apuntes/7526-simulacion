%% simulacion.tex
%% V1
%% 2016/08/27
%% por FIUBA Apuntes
%% http://fiuba-apuntes.github.io/
%%
%% Esta es la clase a usar para los documentos de FIUBA Apuntes.
%%
%% Licencia Creative Commons Atribución-NoComercial-CompartirIgual 4.0
%%
%%
\documentclass[12pt, a4paper, twoside]{paquetes-apunte/apunte}

\usepackage[utf8]{inputenc} % Especifica la codificación de caracteres de los documentos.
\usepackage[spanish]{babel} % Indica que el documento se escribirá en español.
\usepackage{subfiles} % Paquete para incluir el preambulo en los sub archivos.

\usepackage{array}
\usepackage{longtable}
\usepackage{float}
\usepackage{multirow}

\makeatletter

%%%%%%%%%%%%%%%%%%%%%%%%%%%%%% LyX specific LaTeX commands.
%% Because html converters don't know tabularnewline
\providecommand{\tabularnewline}{\\}

%%%%%%%%%%%%%%%%%%%%%%%%%%%%%% User specified LaTeX commands.

\makeatother

%% *** LISTINGS PACKAGE ***

\usepackage{listings}

\lstset{
  literate={á}{{\'a}}1 {é}{{\'e}}1 {í}{{\'i}}1 {ó}{{\'o}}1 {ú}{{\'u}}1 {ñ}{{\~n}}1, % Escapeo caracteres especiales
  basicstyle={\footnotesize\ttfamily},
  emphstyle={\color{red}},
  frame=single,
  keywords={por,cada,para,si},
  keywordstyle={\color{red}},
  numbers=left,
  numbersep=5pt,
  stepnumber=1,
  tabsize=2,
  breaklines=true
}
%\renewcommand{\lstlistingname}{Listado de código}

% Defino la ruta de los paquetes personalizados para el apunte
\newcommand{\rutapaquetes}{./paquetes-apunte}

\usepackage[mostrarlicencia]{\rutapaquetes/caratula} % Caratula personalizada (cargada desde caratula.sty)
\usepackage[ocultarrevisores]{\rutapaquetes/colaboradores} % Seccion de colaboradores (cargada y creada con colaboradores.sty)
\usepackage{\rutapaquetes/historial} % Seccion de historial de cambios (cargada y creada con historial.sty)

%% *** FIUBA APUNTES MACROS PACKAGE ***

\usepackage{\rutapaquetes/macros} % Macros útiles para los apuntes (cargado desde macros.sty)

% Defino los entornos de los cuadros
%\cuadro{definicion}{gray}
\cuadro{bloque}{gray}

%% *** CONFIGURACION ***

\newcommand{\imgdir}{../resources/images} % Ruta de las imágenes

% Define los directorios de las imágenes y gráficos
\graphicspath{ {\imgdir/} {\rutapaquetes/} }

\nombremateria{Simulación (75.26)}

% Configura la caratula
\materia{Simulación (75.26)}
\tipoapunte{Resumen}
%\tema{Tema de la Materia}
%\subtema{Subtema}

\begin{document}

\maketitle

\tableofcontents

\newpage
\section{Definiciones}

\paragraph{Sistema:} conjunto de elementos que interactúan entre si,
que se aíslan del universo para su estudio.

\paragraph{Modelo:} representación abstracta de un sistema real.

\begin{itemize}
  \item Se puede demostrar su \textbf{invalidez} con una no concordancia entre
    los resultados obtenidos en el modelo y los obtenidos en el sistema real.

  \item Deben estar identificadas las \textbf{entidades} y sus atributos:
  \begin{description}
    \item[Entidades permanentes] \emph{savevalues}, \emph{functions},
      \emph{variables}, \emph{SNAs}, \emph{facilities}, \emph{storages},
      \emph{queues}, \emph{tables}
    \item[Entidades transitorias] transacciones
  \end{description}

  \item Debe incluir \textbf{variables}:
%  \begin{description}
%    \item[Endógenas] Internas, controladas por el sistema
%    \item[Exógenas] Externas, no controladas por el sistema
%  \end{description}
%  \begin{description}
%    \item[Cualitativas] Preferencia personal
%    \item[Cuantitativas] Frecuencia con que arriban clientes
%  \end{description}

  \begin{table}[h]
    \centering
    \begin{tabular}{|c|c|}
      \hline
      \textbf{Endógenas} & \textbf{Exógenas} \\
      \hline
      Internas, controladas por el sistema & Externas, no controladas por el sistema \\
      \hline
      \hline
      \textbf{Cualitativas} & \textbf{Cuantitativas} \\
      \hline
      Preferencia personal & Frecuencia con que arriban clientes \\
      \hline
    \end{tabular}
    \caption{Tipos de variables}
  \end{table}

  \item Compuesto por \textbf{bloques} que representan las acciones que realizan
    las \textbf{transacciones} en el sistema.
\end{itemize}

%\paragraph{Tipos de modelos:}
%\begin{description}
%  \item[Numérico] Se conoce instante a instante el valor de las variables, pero no la
%  relación que hay entre ellas
%  \item[Analítico] Se conoce la relación entre las variables
%\end{description}

%\begin{description}
%  \item[Matemático] Representación abstracta
%  \item[Físico] Representación tangible
%\end{description}

%\begin{description}
%  \item[Dinámico] Su estado varía en el tiempo
%  \item[Estático] Su estado no varía en el tiempo
%\end{description}

%\begin{description}
%  \item[Discreto] Los cambios se producen de a saltos
%  \item[Continuo] Los cambios se producen gradualmente
%\end{description}

%\begin{description}
%  \item[Estocástico] Existe azar en la ocurrencia de eventos
%  \item[Determinístico] No existe azar en la ocurrencia de eventos
%\end{description}

\begin{table}[h]
  \centering
  \begin{tabular}{ | >{\centering}p{0.4\textwidth} | >{\centering}p{0.4\textwidth} | }
    \hline
    \textbf{Numérico} & \textbf{Analítico} \tabularnewline
    \hline
    Se conoce instante a instante el valor de las variables, pero no la
    relación que hay entre ellas & Se conoce la relación entre las variables \tabularnewline
    \hline
    \hline
    \textbf{Matemático} & \textbf{Físico} \tabularnewline
    \hline
    Representación abstracta & Representación tangible \tabularnewline
    \hline
    \hline
    \textbf{Dinámico} & \textbf{Estático} \tabularnewline
    \hline
    Su estado varía en el tiempo & Su estado no varía en el tiempo \tabularnewline
    \hline
    \hline
    \textbf{Discreto} & \textbf{Continuo} \tabularnewline
    \hline
    Los cambios se producen de a saltos & Los cambios se producen gradualmente \tabularnewline
    \hline
    \hline
    \textbf{Estocástico} & \textbf{Determinístico} \tabularnewline
    \hline
    Existe azar en la ocurrencia de eventos & No existe azar en la ocurrencia de eventos \tabularnewline
    \hline
  \end{tabular}
  \caption{Tipos de modelos}
\end{table}

\paragraph{Simular:} ensayar en un modelo una alternativa para inferir lo que
pasaría en el sistema real si se aplica dicha alternativa.
Es predecir el futuro ante hipótesis ciertas.

\begin{itemize}
  \item Razones por las que puede fallar una simulación:
  \begin{enumerate}
    \item Modelo inválido, no representa fielmente al sistema real
    \item Las alternativas simuladas no son buenas
  \end{enumerate}

  \item Objetivo: predicción

  \item Fases:
  \begin{enumerate}
    \item Definir el sistema (aislar las partes a estudiar)
    \item Construir el modelo (representar el sistema)
    \item Simular (ensayar alternativas)
    \item Sacar conclusiones
    \item Estudiar las conclusiones y aconsejar la mejor alternativa
  \end{enumerate}

  \item Ventajas:
  \begin{itemize}
    \item Permite adquirir una rápida experiencia a bajo costo y sin riesgos,
    \item Permite identificar áreas con problemas en sistemas complejos,
    \item Permite un estudio sistemático de alternativas,
    \item Es ilimitado en cuanto a complejidad.
  \end{itemize}

  \item Desventajas:
  \begin{itemize}
    \item No supera a las técnicas analíticas,
    \item No se puede asegurar que el modelo es válido,
    \item No existe un criterio científico de selección de la estrategia de
      simulación,
    \item Existe un riesgo de utilizar un modelo fuera de los límites para el
      cual fue construido.
  \end{itemize}
\end{itemize}

\paragraph{Estrategia de simulación:} conjunto de alternativas que se
definen para un ensayo.

\paragraph{Evento:} acción que provoca un cambio de estado, modificando
los atributos de las entidades.


\section{Cadenas}

Las transacciones son temporariamente ligadas a entidades ocupando
listas vinculadas llamadas cadenas.
\begin{itemize}
  \item La \textbf{Cadena de Eventos Futuros }(FEC) es una cadena ordenada
    en el tiempo que retiene transacciones que deben esperar por un tiempo
    de simulación. Cuando toda actividad de la simulación durante el tiempo
    de reloj corriente es completada, la siguiente transacción es tomada
    de la FEC. Los bloques ADVANCE y GENERATE son la única manera de colocar
    una transacción en la FEC.


  \item La \textbf{Cadena de Eventos Corrientes} (CEC) es una lista encadenada
    de transacciones preparadas que ya tienen bloques para ser ingresados
    antes que la simulación avance.Cuando la transacción activa descansa
    en alguna cadena, la primer transacción de mayor prioridad en la CEC
    se vuelve la transacción activa. Si la CEC esta vacía, la transacción(s)
    más inminente en la Cadena de Eventos Futuros (FEC) es o son movidas
    a la CEC. Esta acción siempre causa un avance en el sistema de reloj.

    La ubicación de la CEC puede ser modificada por los bloques PRIORITY
    y BUFFER.

  \item Cada entidad tiene una cadena de transacciones bloqueadas, llamada
    \textbf{Cadena de Reintento} (RETRY CHAIN). Las transacciones que
    fallan todos los testeos requeridos para entrar a un bloque son colocadas
    en una Cadena de Reintento. Estos testeos ocurren cuando una transacción
    intenta entrar en un bloque GATE o TEST.

  \item Las \textbf{Cadenas de Usuario} (USER CHAIN) son listas encadenadas
    de transacciones que han sido removidas de la Cadena de Eventos Corrientes
    por un bloque LINK.
\end{itemize}

\section{Funcionamiento del GPSS}

\begin{lstlisting}
por cada GENERATE del modelo:
  crea una transacción
  pone la transacción en la FEC (Future Event Chain)

mientras (cantidad de transacciones destruidas < parámetro n del START)
    ordena la FEC en orden cronológico

    primer evento := FEC [0]
    reloj := hora del primer evento

    para cada transacción de la FEC:
      si (hora de transacción == reloj)
        agregar transacción a CEC (Current Event Chain)

    ordenar la CEC por prioridad

    para cada transacción de la CEC:
      mover la transacción por el modelo hasta que:
        si (transacción ejecuta un LINK):
          transacción queda esperando
        si (transacción quiere tomar un recurso no disponible):
          poner transacción en DELAY CHAIN del recurso
        si (transacción atraviesa una condición falsa sin alternativas):
          poner transacción en RETRY CHAIN de la condición
        si (transacción quiere arrebatar un recurso y no puede):
          poner transacción en PENDING CHAIN del recurso
        si (transacción quiere hacer ADVANCE con tiempo != 0):
          poner transacción en FEC con el instante de finalización
        si (transacción muere):
          quitar transacción de CEC
        si (transacción es la primera de un ASSEMBLE):
          transacción queda esperando
        si (transacción genera nueva transacción X)
          poner X en CEC
\end{lstlisting}


\section{Bloques y objetos}

\paragraph{Bloque:} representa las acciones que realizan las transacciones
en el sistema.

\begin{itemize}
  \item Objetos transitorios
  \begin{itemize}
    \item \textbf{Transacción:} elemento que se mueve por el modelo.
    \begin{itemize}
      \item Atributos: se instancian la primera vez que se les asigna valor.
        (\texttt{ASSIGN}, \texttt{SELECT}, \texttt{MARK}, \texttt{ALTER}).
      \begin{itemize}
        \item \textbf{Prioridad}: 0 (mínima), 100 (máxima). Se puede alterar con
          los bloques \texttt{PRIORITY} y \texttt{ALTER}, y examinar con el
          bloque \texttt{SCAN}. Se inicializa en \texttt{GENERATE.}
        \item \texttt{\textbf{M1}}: edad, se inicializa en cero al nacer la transacción,
          y se puede alterar con el bloque \texttt{MARK}. Al momento de consultar
          el \texttt{M1}, se hace la resta entre la hora actual y la hora de
          nacimiento.
        \item \textbf{Parámetros}: son locales a la transacción. Se modifican con
          \texttt{ASSIGN} y \texttt{SELECT}.
      \end{itemize}

      \item Cada transacción tiene un \textbf{puntero} que apunta al primer primer
        miembro de la familia, que, al nacer cada transacción, apunta a sí
        misma (cada transacción creada por \texttt{GENERATE} pertenece a familias
        independientes entre sí).
    \end{itemize}
  \end{itemize}

  \item Objetos permanentes:
  \begin{itemize}
    \item Recursos definidos explícitamente: \texttt{STORAGE}, \texttt{TABLE},
      \texttt{QTABLE}.
    \item Recursos definidos implícitamente: \texttt{FACILITY}, \texttt{QUEUE}.
  \end{itemize}
\end{itemize}

Bloques que generan transacciones: \texttt{GENERATE}, \texttt{SPLIT}.

Bloques que colocan transacciones en la FEC (\emph{Future Event Chain}):
\texttt{GENERATE}, \texttt{ADVANCE}.

Bloques que destruyen transacciones: \texttt{TERMINATE}, \texttt{ASSEMBLE}.

Bloques que bloquean transacciones si no existe rótulo alternativo
y la proposición es falsa: \texttt{TEST}, \texttt{GATE}.

~\\

SNAs:
\begin{itemize}
  \item \texttt{AC1}: hora absoluta, en segundos. Se inicializa en cero al
    comenzar la simulación. Es el tiempo simulado desde el último \emph{clear}.
  \item RN3, RN4, RN5...: número al azar entre 0 y 999.
  \item C1: hora relativa, en segundos. Es el tiempo simulado desde el último
    \emph{reset}.
\end{itemize}

\begin{table}[h]
  \centering
  \begin{tabular}{| >{\centering}p{0.5\textwidth} | >{\centering}p{0.4\textwidth} |}
    \hline
    \textbf{Facility} & \textbf{Storage} \tabularnewline
    \hline
    \hline
    \texttt{SEIZE, RELEASE} & \texttt{ENTER, LEAVE} \tabularnewline
    \hline
    Capacidad: 1 & Capacidad: $n$ \tabularnewline
    \hline
    Es obligatorio que la transacción que ejecuta el \texttt{RELEASE}
    sea la que anteriormente ejecutó \texttt{SEIZE}. &
    No es obligatorio que la transacción que ejecuta el \texttt{LEAVE}
    sea la que anteriormente ejecutó el \texttt{ENTER}. \tabularnewline
    \hline
    Es arrebatable. & No es arrebatable. \tabularnewline
    \hline
    Cadena de Demora (DELAY CHAIN) \textendash{} Una cadena de prioridad
    de transacciones esperando por la propiedad de una \emph{facility}.

    Cadena de Pendientes (PENDING CHAIN) \textendash{} Una lista de transacciones
    esperando por apropiarse de la \emph{facility}.

    Cadena de Interrupción (INTERRUPT CHAIN) \textendash{} Una lista de
    transacciones que han sido apropiadas desde la propiedad de la \emph{facility}.

    Cadena de Reintento (RETRY CHAIN) \textendash{} Una lista de transacciones
    que están esperando por el cambio de estado de una \emph{facility}. &

    Cadena de Demora (DELAY CHAIN) \textendash{} Una cadena de prioridad
    de transacciones esperando por una unidad \emph{storage}.

    Cadena de Reintento (RETRY CHAIN) \textendash{} Una lista de transacciones
    que están esperando por el cambio de estado del \emph{storage}. \tabularnewline
    \hline
  \end{tabular}
  \caption{Diferencia entre \emph{storages} y \emph{facilities}}
\end{table}

\begin{table}[h]
  \centering
  \begin{tabular}{|c|c|c|c|}
    \hline
    & \textbf{Parámetro} & \textbf{\emph{Savevalue}} & \textbf{\emph{Logic switch}} \tabularnewline
    \hline
    \hline
    ¿Localidad? & Pertenece a una transacción & Global & Global \tabularnewline
    \hline
    ¿Valor posible? & Cualquier valor & Cualquier valor & 2 valores - set (1), reset (0) \tabularnewline
    \hline
    ¿Valor inicial? & - & 0 & reset (0) \tabularnewline
    \hline
  \end{tabular}
  \caption{Diferencia entre parámetros, \emph{savevalues} y \emph{logic switches}}
\end{table}

\newpage

\section{Cadenas de usuario}

\paragraph{Cadena:} objeto donde las transacciones se ``encadenan''
de modo tal que si otra transacción no las libera, jamás saldrán de
allí.

Permiten representar interacciones entre dos subsistemas.

\begin{itemize}
  \item Transporte de transacciones: la transacción \textbf{activa} realiza
    el transporte, y la transacción \textbf{pasiva} es transportada.
  \item Recursos que no siempre están disponibles: se los representa como
    una transacción y como una \emph{facility}.
  \item Impaciencia de clientes: se subdivide al cliente en dos partes; uno
    es el original, y la copia es la transacción que espera un tiempo fijo.
  \begin{itemize}
    \item Impaciencia de grado 1: por ejemplo, si la cantidad de gente en cola
      supera 8, me voy de la cola.
    \item Impaciencia de grado 2: por ejemplo, si el tiempo que pasé en cola
      supera 1 hora, me voy de la cola.
  \end{itemize}
\end{itemize}

\begin{lstlisting}
        GENERATE    clientes
        TEST  LE    Q$JUAN,7,CHAU       ;si en la cola de Juan hay mas de 7 clientes, va a CHAU
        SAVEVALUE   CUEN+,1
        ASSIGN      4,X$CUEN            ;P4 tiene el numero de cliente
        SPLIT       1,IMPA              ;generamos una copia y la mandamos a IMPA
        QUEUE       JUAN                ;hace cola para el facility Juan
        LINK        JUAN,FIFO
ATIEN   SEIZE       JUAN
        DEPART      JUAN
        ADVANCE     tiempo de atencion
        RELEASE     JUAN
        TERMINATE
IMPA    ADVANCE     tiempo de paciencia ;soy la copia que espera un tiempo
        UNLINK      JUAN,SEVA,1,4,*4    ;me impacienté: desencadeno a 1 transacción cuyo P4 sea igual al P4 mío
        TERMINATE                       ;destruyo la copia
SEVA    DEPART      JUAN
CHAU    TERMINATE
\end{lstlisting}


\newpage

\section{Cartilla de bloques}

\paragraph{Referencias}
\begin{description}
  \item[ID] número o nombre
\end{description}

\newcommand{\bloquetitle}[1]{\texttt{#1}}
\newcommand{\bloquesintax}[1]{\paragraph{Bloque} \texttt{#1} \\}
\newcommand{\accion}[1]{\paragraph{Acción que representa} #1 \\}
\newcommand{\parametros}[1]{\paragraph{Parámetros} #1}
\newcommand{\paramtitle}[1]{\texttt{#1}:}
\newcommand{\parametrosdefault}[1]{\paragraph{Parámetros por defecto} #1}

\begin{bloque}{\bloquetitle{START}}
  \bloquesintax{START n}
  \accion{Simular el modelo hasta contabilizar $n$ transacciones terminadas}
  \parametros{
    \begin{description}
      \item[\paramtitle{n}] $\in N$, cantidad de transacciones
    \end{description}
  }
\end{bloque}

\begin{bloque}{\bloquetitle{GENERATE}}
  \bloquesintax{GENERATE [t],[d],[o],[c],[p]}
  \accion{Arribo de transacciones independientes entre sí cada cierto tiempo
    $t \pm d$ o $t \times d$}
  \parametros{
    \begin{description}
      \item[\paramtitle{t}] valor medio entre arribos
      \item[\paramtitle{d}] según el tipo asignado
      \begin{itemize}
        \item Si $d \in N$, es el desvío máximo permitido.
        \item Si $d$ es una \texttt{FUNCTION}, se multiplica $t \times d$ para
          calcular el próximo nacimiento
      \end{itemize}
      \item[\paramtitle{o}] instante en que se genera la primera transacción
      \item[\paramtitle{c}] cantidad máxima de transacciones a generar
      \item[\paramtitle{p}] prioridad de la transacción a generar
    \end{description}
  }
  \parametrosdefault{
    \begin{itemize}
      \item $t=0$
      \item $d=0$
      \item $o=1$
      \item $c=\infty$
      \item $p=0$ (prioridad mínima)
    \end{itemize}
  }
\end{bloque}

\begin{bloque}{\bloquetitle{ADVANCE}}
  \bloquesintax{ADVANCE t[,d]}
  \accion{Realización de una tarea que dura cierto tiempo $t \pm d$ o $t \times d$}
  \parametros{
    \begin{description}
      \item[\paramtitle{t}] valor medio de la duración de la tarea (constante o SNA)
      \item[\paramtitle{d}] según el tipo asignado
      \begin{itemize}
        \item Si $d \in N$, es el desvío máximo permitido.
        \item Si $d$ es una \texttt{FUNCTION}, se multiplica $t \times d$ para
          calcular la duración. \emph{No puede ser referencia a FUNCTION}.
      \end{itemize}
      \item[] Debe cumplirse que $t\geq d$
    \end{description}
  }
  \parametrosdefault{
    \begin{itemize}
      \item $t=0$
      \item $d=0$
    \end{itemize}
  }
\end{bloque}

\begin{bloque}{\bloquetitle{TERMINATE}}
  \bloquesintax{TERMINATE [n]}
  \accion{Salida de $n$ transacciones. Si $n$ no se especifica, no se contabiliza
    la transacción pero la misma se destruye.}
  \parametros{
    \begin{description}
      \item[\paramtitle{n}] $\in N$
    \end{description}
  }
  \parametrosdefault{
    \begin{itemize}
      \item $n=0$
    \end{itemize}
  }
\end{bloque}

\begin{bloque}{\bloquetitle{SEIZE}}
  \bloquesintax{SEIZE f}
  \accion{Tomar un recurso de uso exclusivo (\emph{facility})}
  \parametros{
    \begin{description}
      \item[\paramtitle{f}] identificador de la \emph{facility}
    \end{description}
  }
\end{bloque}

\begin{bloque}{\bloquetitle{RELEASE}}
  \bloquesintax{RELEASE f}
  \accion{Dejar un recurso de uso exclusivo (\emph{facility})}
  \parametros{
    \begin{description}
      \item[\paramtitle{f}] identificador de la \emph{facility}
    \end{description}
  }
\end{bloque}

\begin{bloque}{\bloquetitle{STORAGE}}
  \bloquesintax{IDSTORAGE STORAGE n}
  \accion{Definir la capacidad de un \emph{storage} en $n$ transacciones}
  \parametros{
    \begin{description}
      \item[\paramtitle{IDSTORAGE}] identificador del \emph{storage}
      \item[\paramtitle{n}] $\in N$, $n < 30000$
    \end{description}
  }
\end{bloque}

\begin{bloque}{\bloquetitle{ENTER}}
  \bloquesintax{ENTER s[,l]}
  \accion{Intentar ocupar uno o más lugares de un recurso de uso compartido
    (\emph{storage})}
  \parametros{
    \begin{description}
      \item[\paramtitle{s}] identificador del \emph{storage}
      \item[\paramtitle{l}] lugares que ocupa la transacción
    \end{description}
  }
  \parametrosdefault{
    \begin{itemize}
      \item $l=1$
    \end{itemize}
  }
\end{bloque}

\begin{bloque}{\bloquetitle{LEAVE}}
  \bloquesintax{LEAVE s[,l]}
  \accion{Dejar uno o más lugares de un recurso de uso compartido (\emph{storage}).
    No es necesario haber ejecutado \texttt{ENTER} previamente.}
  \parametros{
    \begin{description}
      \item[\paramtitle{s}] identificador del \emph{storage}
      \item[\paramtitle{l}] lugares que deja la transacción
    \end{description}
  }
  \parametrosdefault{
    \begin{itemize}
      \item $l=1$
    \end{itemize}
  }
\end{bloque}

\begin{bloque}{\bloquetitle{TRANSFER}}
  \bloquesintax{TRANSFER [p,[a]],b}
  \accion{Bifurcar estocásticamente. Con probabilidad $p$ se bifurca a $b$,
    y con probabilidad $1-p$ se bifurca a $a$. Si $a$ no se especifica,
    bifurca al bloque siguiente con probabilidad $1-p$. Si ni $p$ ni $a$
    se especifican, bifurca siempre a $b$}
  \parametros{
    \begin{description}
      \item[\paramtitle{p}] $0 \leq p < 1$, o $p$ es una referencia a una función
        de $n$ puntos tal que $0 \leq y_{i} < 1000$ $(i=1..n)$
      \item[\paramtitle{a}] rótulo o referencia a función
      \item[\paramtitle{b}] rótulo o referencia a función
    \end{description}
  }
  \parametrosdefault{
    \begin{itemize}
      \item $p=1$
    \end{itemize}
  }
\end{bloque}

\begin{bloque}{\bloquetitle{TRANSFER}}
  \bloquesintax{TRANSFER SIM,b,c}
  \accion{Si el \emph{delay indicator} de la transacción está en ON (set)
    \textemdash si alguno de los \texttt{GATE/TEST} anteriores no se cumple\textemdash,
    la transacción es enviada al rótulo $c$ y pone el \emph{delay indicator}
    en OFF (reset). Si el \emph{delay indicator} está en OFF \textemdash todos
    los \texttt{GATE/TEST} anteriores se cumplen \textbf{simultáneamente}\textemdash,
    la transacción es enviada al rótulo $b$.}
  \parametros{
    \begin{description}
      \item[\paramtitle{b}] rótulo o referencia a función
      \item[\paramtitle{c}] rótulo o referencia a función
    \end{description}
  }
\end{bloque}

\begin{longtable}{|lc>{\raggedright}p{3cm}|>{\raggedright}p{5cm}|>{\raggedright}p{7cm}|>{\raggedright}p{4cm}|}
\hline
\multicolumn{3}{|l|}{\textbf{Bloque}} & \textbf{Acción que representa} & \textbf{Parámetros} & \textbf{Default}\tabularnewline
\hline
\hline
 & \texttt{QUEUE} & \texttt{f{[},l{]}} & Hacer cola para la \emph{facility} $f$ & \begin{itemize}
\item $f=$ id de la \emph{queue}
\item $l=$ lugares que ocupa la transacción\end{itemize}
 & $l=1$\tabularnewline
\hline 
 & \texttt{DEPART } & \texttt{f{[},l{]}} & Salir de la cola de la \emph{facility} $f$, si la \texttt{QTABLE}
respectiva está definida, registra allí las estadísticas & \begin{itemize}
\item $f=$ id de la \emph{queue}
\item $l=$ lugares que deja la transacción \end{itemize}
 & $l=1$\tabularnewline
\hline 
\texttt{NTABLE} & \texttt{TABLE} & \texttt{v,x$_{0}$,$\Delta$x,n} & Definir una tabla de distribución de frecuencia. Necesita de \texttt{TABULATE}. & \begin{itemize}
\item $v=$ valor a tabular de la transacción que ejecuta el \emph{tabulate}
(cualquier SNA)
\item $x_{0}=$ Límite superior del primer intervalo
\item $\Delta x=$ tamaño de cada intervalo
\item $n=$cantidad de intervalos\end{itemize}
 & \tabularnewline
\hline 
 & \texttt{TABULATE} & \texttt{t} & Agrega un valor a una \texttt{TABLE} & $t=$ id de la \texttt{TABLE} & \tabularnewline
\hline 
\texttt{NQTABLE} & \texttt{QTABLE} & \texttt{ntable,t$_{0}$,$\Delta t$,n} & Define una tabla de distribución de frecuencias para tiempos en cola.
No necesita de \texttt{TABULATE}. & \begin{itemize}
\item $NQTABLE=$ id de la\emph{ queue}
\item $t_{0}=$ límite superior del primer intervalo
\item $\Delta t=$ tamaño de cada intervalo
\item $n=$ cantidad de intervalos\end{itemize}
 & \tabularnewline
\hline 
 & \texttt{ASSIGN} & \texttt{p,v}

\texttt{p+,v}

\texttt{p-,v} & Asigna, suma o resta el valor $v$ al parámetro $p$ de la transacción
activa & $p=$ id del parámetro (puede ser un SNA)

$v=$ valor a asignar, sumar o restar (puede ser una constante o un
SNA) & \tabularnewline
\hline 
\texttt{SELECT} & \texttt{op logico}\footnote{\texttt{Logic Switch: LS, LR}

\texttt{Facility: U, NU, I, NI, FV, FNV}

\texttt{Storage: SE, SNE, SF, SNF, SV, SNV}} & \texttt{p,v$_{1}$,v$_{2}${[},,,f{]}} & Busca una \emph{facility, storage }o\emph{ logic switch} según el
\texttt{op logico} y la almacena en el parámetro $p$ de la transacción
activa. Al finalizar, $v_{1}\leq a\leq v_{2}$ o, si no pudo seleccionar,
bifurca a $f$. Si $f$ no esta, establece que $p=0$ y continúa en
el bloque siguiente & \begin{itemize}
\item $p=$ id del parámetro
\item $v_{1}=$ número de objeto menor desde el cual se hace el \texttt{SELECT}
\item $v_{2}=$ número de objeto mayor desde el cual se hace el \texttt{SELECT}
\item $f=$ rótulo al que bifurca si la selección fue infructuosa\end{itemize}
 & \tabularnewline
\hline 
\texttt{SELECT} & \texttt{op relacion}\footnote{\texttt{EQ, L, LE, G, GE, NE}} & \texttt{a,v$_{1}$,v$_{2}$,d,e{[},f{]}} & Busca una objeto $e$ según el \texttt{op relacion }y lo almacena
en el parámetro $a$ de la transacción activa. Al finalizar, $v_{1}\leq a\leq v_{2}$
o, si no hay desocupadas, bifurca a $f$. Si $f$ no esta, establece
que $a=0$ y continúa en el bloque siguiente & \begin{itemize}
\item $a=$ id del parámetro
\item $v_{1}=$ número de objeto menor desde el cual se hace el \texttt{SELECT}
\item $v_{2}=$ número de objeto mayor desde el cual se hace el \texttt{SELECT}
\item $d=$ valor con el que se compara
\item $e=$ clase de objeto que se compara (SNAs)
\item $f=$ rótulo al que bifurca si la selección fue infructuosa\end{itemize}
 & \tabularnewline
\hline 
\texttt{SELECT} & \texttt{MIN/MAX} & \texttt{p,v$_{1}$,v$_{2}$,,e} & Busca un\emph{ facility }desocupado según el criterio $e$ y lo almacena
en $p$. Si el criterio no alcanza para definir, $p=v_{1}$ & \begin{itemize}
\item $p=$ id del parámetro
\item $v_{1}=$ número menor del objeto a consultar
\item $v_{2}=$ número mayor del objeto a consultar
\item $e=$ clase de objeto en la que se busca el mínimo o máximo\end{itemize}
 & \tabularnewline
\hline 
\texttt{NVARIABLE} & \texttt{VARIABLE} & \texttt{f} & Define una función que se evalúa cada vez que una transacción hace
referencia a ella. Se truncan los decimales de los resultados intermedios
y del resultado final. & $f=$ función con operadores (cualquier SNA)

\noindent \centering{}%
\begin{tabular}{|c|c|}
\hline 
+ & suma\tabularnewline
\hline 
\hline 
- & resta\tabularnewline
\hline 
\hline 
\# & multiplicación\tabularnewline
\hline 
\hline 
/ & división\tabularnewline
\hline 
\hline 
@ & módulo\tabularnewline
\hline 
\hline 
\textasciicircum{} & potencia\tabularnewline
\hline 
\end{tabular} & \tabularnewline
\hline 
\texttt{NFUNCTION} & \texttt{FUNCTION} & \texttt{x,tn}

\texttt{x$_{1}$,y$_{1}$/x$_{2}$,y$_{2}$/.../x$_{n}$,y$_{n}$} & Devuelve un valor cada vez que se la invoca. Define una función de
$n$ puntos. Si $x\leq x_{1}$, devuelve $y_{1}$, si $x\geq x_{n}$,
devuelve $y_{n}$. Se truncan los decimales de los resultados intermedios
y del resultado final. & \begin{itemize}
\item $x=$ variable independiente (cualquier SNA). Si $x$ es RN, $0<x_{i}\leq1$
\item $t=$ tipo de la función ($C=$continua, $D=$discreta, $E=$discreta
de atributos numéricos)\footnote{Debe cumplirse que $x_{1}\leq x_{2}\leq...\leq x_{n}$.
Si es \textbf{continua}, y $x_{i}<x<x_{i+1}$ se interpola linealmente
entre $y_{i}$ e $y_{i+1}$. Si es \textbf{discreta}, y $x_{i}<x<x_{i+1}$,
se devuelve $y_{i+1}$ (los $y$ pueden ser rótulos). Si es \textbf{discreta
de atributos numéricos}, es como la discreta pero las $y$ deben ser
SNAs.}
\item $n=$ cantidad de puntos que definen la función\end{itemize}
 & \tabularnewline
\hline 
 & \texttt{SAVEVALUE} & \texttt{s,v}

\texttt{s+,v}

\texttt{s-,v} & Asigna, suma o resta el valor $v$ al \emph{savevalue} $s$ & \begin{itemize}
\item $s=$ id del \emph{savevalue}
\item $v=$ valor a asignar, sumar o restar\end{itemize}
 & \tabularnewline
\hline 
 & \texttt{INITIAL} & \texttt{s{[},a{]}} & Asigna el valor inicial $a$ al \emph{savevalue} $s$ & $s=$ \textbf{referencia} al \emph{savevalue} & ?\tabularnewline
\hline 
 & \texttt{LOOP} & \texttt{p,a} & Resta 1 al parámetro $p$. Si $p>0$, bifurca al rótulo $a$. Si $p=0$,
continúa en el bloque siguiente & \begin{itemize}
\item $p=$ id del parámetro a decrementar
\item $a=$ rótulo al que bifurca\end{itemize}
 & \tabularnewline
\hline 
\texttt{NMATRIX} & \texttt{MATRIX} & \texttt{,f,c} & Define una matriz & \begin{itemize}
\item $f=$ cantidad de filas
\item $c=$ cantidad de columnas\end{itemize}
 & \tabularnewline
\hline 
 & \texttt{MSAVEVALUE} & \texttt{m,f,c,v}

\texttt{m+,f,c,v}

\texttt{m-,f,c,v} & Asigna, suma o resta el valor $v$ del elemento $m[f,c]$ & \begin{itemize}
\item $m=$ id de la matriz
\item $f=$ fila (cualquier SNA)
\item $c=$ columna (cualquier SNA)
\item $v=$ valor a asignar, sumar o restar\end{itemize}
 & \tabularnewline
\hline 
\texttt{TEST} & \texttt{op} & \texttt{a,b{[},f{]}} & Si $a\,op\,b$ es V, continúa en el bloque siguiente. Si es F, bifurca
al rótulo $f$; si es F y no se especifico $f$, bloquea la transacción
hasta que sea V & \begin{itemize}
\item $op$ puede ser $e$, $ne$, $g$, $ge$, $l$, $le$
\item $a$ y $b$ pueden ser constantes o SNAs
\item $f=$ rótulo al que bifurca\end{itemize}
 & \tabularnewline
\hline 
 & \texttt{SPLIT} & \texttt{c,r{[},n{]}} & Genera $c$ nuevas transacciones, \textquotedblleft clonando\textquotedblright{}
la transacción activa (M1, PR, parámetros). Las copias bifurcan a
$r$ y el original continúa en el bloque siguiente & \begin{itemize}
\item $c=$ cantidad de copias (constante o SNA)
\item $r=$ rótulo al que bifurcan las copias
\item $n=$ id de parámetro a utilizar para enumerar el original y las copias\end{itemize}
 & \tabularnewline
\hline 
 & \texttt{ASSEMBLE} & \texttt{a} & Reúne un grupo de transacciones y deja pasar solo una transacción
que representa al grupo, las restantes $a-1$ son destruidas & $a=$ cantidad de miembros de la familia a sincronizar & \tabularnewline
\hline 
 & \texttt{GATHER} & \texttt{m} & Reúne un grupo de transacciones y deja pasar a todas una vez reunida
la cantidad especificada, sin destruir ninguna transacción & $m=$ cantidad de miembros de la familia a reunir & \tabularnewline
\hline 
 & \texttt{MARK} & \texttt{{[}p{]}} & Se almacena un M1 ``paralelo'' en el parámetro $p$ de la transacción.
Si $p$ no se especifica, se inicializa M1 en cero. & $p=$ id del parámetro & $p=M1$\tabularnewline
\hline 
 & \texttt{PRIORITY} & \texttt{p} & Cambia la prioridad de la transacción activa & $p=$ nueva prioridad & \tabularnewline
\hline 
\texttt{LOGIC} & \texttt{v} & \texttt{ll} & Cambia el estado del \emph{logic switch} $ll$ por el valor de $v$ & \begin{itemize}
\item $v=$ S (set), R (reset), I (invertido)
\item $ll=$ id de la llave\end{itemize}
 & \tabularnewline
\hline 
 & \texttt{INITIAL} & \texttt{rll{[},v{]}} & Inicializa el estado de la llave $rll$ con el valor $v$. & \begin{itemize}
\item $rll=$\textbf{ referencia} a una llave lógica
\item $v=$ 1 (set), 0 (reset)\end{itemize}
 & $v=1$(set)\tabularnewline
\hline 
\texttt{GATE} & \texttt{op lógico\ref{op_logicos}} & \texttt{a{[},f{]}} & Consulta el estado del \emph{logic switch}, \emph{facility}, \emph{storage}
o rótulo $a$. Si $op\,A$ es V, continúa en el bloque siguiente.
Si es F, bifurca a $f$. Si no se especifica $f$, la transacción
se bloquea hasta que sea V (\emph{delay indicator }en ON) & \begin{itemize}
\item $a=$ id del \emph{logic switch}, \emph{facility}, \emph{storage}
\item $f=$ rótulo al que bifurca si la condición es falsa.\end{itemize}
 & \tabularnewline
\hline 
\texttt{COUNT} & \texttt{op logico}\footnote{\label{op_logicos}Logic Switch: LS, LR

Facility: U, NU, I, NI, FV, FNV

Storage: SE, SNE, SF, SNF, SV, SNV} & \texttt{a,v$_{1}$,v$_{2}$} & Cuenta la cantidad de objetos desde $v_{1}$ hasta $v_{2}$ que cumplen
la condición $op\,objeto$, y lo almacena en $a$. & \begin{itemize}
\item $a=$ parámetro que recibe la cantidad de objetos que cumplen la condición
\item $v_{1}=$ número menor del objeto a consultar
\item $v_{2}=$ número mayor del objeto a consultar\end{itemize}
 & \tabularnewline
\hline 
\texttt{COUNT} & \texttt{op relacion}\footnote{\texttt{EQ, L, LE, G, GE, NE}} & \texttt{a,v$_{1}$,v$_{2}$,d,e} & Cuenta la cantidad de objetos desde $v_{1}$ hasta $v_{2}$ que cumplen
la condición $objeto\,op\,d$, y lo almacena en $a$. & \begin{itemize}
\item $a=$ parámetro que recibe la cantidad de objetos que cumple la condición
\item $v_{1}=$ número menor del objeto a consultar
\item $v_{2}=$ número mayor del objeto a consultar
\item $d=$ valor con el que se compara
\item $e=$ clase de objeto que se selecciona (SNA)\end{itemize}
 & \tabularnewline
\hline 
\texttt{COUNT} & \texttt{MIN/MAX} & \texttt{a,v$_{1}$,v$_{2}$,,e} & Cuenta la cantidad de objetos desde $v_{1}$ hasta $v_{2}$ que son
mínimo o máximo, y lo almacena en $a$. & \begin{itemize}
\item $a=$ parámetro que recibe la cantidad de objetos que son mínimo o
máximo
\item $v_{1}=$ número menor del objeto a consultar
\item $v_{2}=$ número mayor del objeto a consultar
\item $e=$ clase de objeto que se compara (SNA)\end{itemize}
 & \tabularnewline
\hline 
 & \texttt{JOIN} & \texttt{g} & La transacción activa se une al grupo $g$ ``personalmente''. & $g=$ id del grupo. & \tabularnewline
\hline 
 & \texttt{EXAMINE} & \texttt{g,,f} & Verifica si la transacción activa pertenece al grupo $g$. Si pertenece,
continúa en el bloque siguiente. Si no, bifurca a $f$. & \begin{itemize}
\item $g=$ id del grupo.
\item $f=$ rótulo al que bifurca.\end{itemize}
 & \tabularnewline
\hline 
 & \texttt{REMOVE} & \texttt{g} & La transacción activa se elimina al grupo $g$ ``personalmente''. & $g=$ id del grupo. & \tabularnewline
\hline 
\texttt{REMOVE} & \texttt{{[}op{]}}\footnote{\texttt{E,G,GE,L,LE,MAX,MIN,NE}} & \texttt{g,{[}c{]},,{[}p{]},{[}v{]},{[}f{]}} & Intenta eliminar $c$ transacciones del grupo $g$ que cumplen la
condición $p\,op\,v$. Si no puede eliminar todas, bifurca a $f$.  & \begin{itemize}
\item $g=$ id del grupo
\item $c=$cantidad de transacciones a eliminar
\item $p=$ idel parámetro que se consulta de la transacción
\item $v=$valor con el que se compara (no se pone si $op=MIN/MAX$)
\item $f=$rótulo al que bifurca si no pudo eliminar $c$ transacciones\end{itemize}
 & $op=E$

$c=ALL$\tabularnewline
\hline 
\texttt{ALTER} & \texttt{{[}op{]}}\footnote{\texttt{E,G,GE,L,LE,NE}} & \multirow{1}{3cm}{\texttt{g,c,p$_{1}$,v$_{1}$,{[}p$_{2}${]},{[}v$_{2}${]},{[}f{]}}} & Intenta alterar $c$ transacciones del grupo $g$ de la siguiente
forma: para cada transacción, si $p_{2}\,op\,v_{2}$ es V, se asigna
$p_{1}=v_{1}$. Si no puede alterar $c$ transacciones, bifurca al
rótulo $f$. & \begin{itemize}
\item $g=$ id del grupo
\item $c=$ cantidad de transacciones a eliminar\footnote{$c$ puede ser \texttt{ALL}}
\item $p_{1}=$ id del parámetro a alterar
\item $v_{1}=$valor a asignar al parámetro $p_{1}$ (constante o SNA)
\item $p_{2}=$ id del parámetro a consultar
\item $v_{2}=$ valor con el que se compara $p_{2}$ (constante o SNA)
\item $f=$rótulo al que bifurca si no pudo alterar $c$ transacciones\end{itemize}
 & $op=E$\tabularnewline
\hline 
\texttt{SCAN} & \texttt{{[}op{]}} & \texttt{g,b,{[}c{]},p$_{1}$,p$_{2}${[},f{]}} & Busca en el grupo $g$ la \textbf{primera} transacción que cumple
la condición $b\,op\,c$. Copia el parámetro $p_{1}$ en el parámetro
$p_{2}$ de la transacción actual. Si no encuentra ninguna, bifurca
a $f$. & \begin{itemize}
\item $OP$ puede ser $e,g,ge,l,le,max,min,ne$
\item $b=$ id del parámetro que se consulta
\item $c=$ valor con el que se compara (si $OP$ es $min/max$, no se pone)
\item $p_{1}=$ id del parámetro de la transacción del grupo que se copia
\item $p_{2}=$ id del parámetro de la transacción actual que recibe el
valor de $p_{1}$\end{itemize}
 & \begin{itemize}
\item $op=E$\end{itemize}
\tabularnewline
\hline 
 & \texttt{LINK} & \texttt{c,r{[},f{]}} & Encadena la transacción pasiva actual a la cadena $c$ según la regla
$r$. Si se especifica el rótulo $f$, la primera transacción \textbf{no
es encadenada} y bifurca al rótulo $f$. & \begin{itemize}
\item $c=$ id de la cadena (constante o SNA)
\item $r=$ regla por la cual se encadena (FIFO, LIFO, PR, Pn, M1)
\item $f=$ rótulo al que va la primera transacción\end{itemize}
 & \tabularnewline
\hline 
\texttt{UNLINK} & \texttt{{[}op{]}} & \texttt{a,b,c,{[}d{]},{[}e{]},{[}f{]}} & La transacción activa actual desencadena $c$ transacciones de la
cadena $a$ tales que se verifique $d\,op\,e$, y las envía al rótulo
$b$. Si no puede liberar todas, bifurca a $f$ (las que fueron liberadas,
quedan liberadas). & \begin{itemize}
\item $op$ puede ser \texttt{E,G,GE,L,LE,NE}
\item $a=$ id de la cadena
\item $b=$ rótulo al que se envían las transacciones liberadas
\item $c=$ cantidad de transacciones que se desencadenan\footnote{$c$ puede ser \texttt{ALL}}
\item $d=$ id de parámetro de la transacción\textbf{ activa} que se consulta
\item $e=$\textbf{ referencia} del parámetro de la transacción \textbf{pasiva}
que se consulta
\item $f=$ rótulo al que bifurca la transacción activa si no pudo liberar
$c$ transacciones.\end{itemize}
 & \begin{itemize}
\item $op=E$\end{itemize}
\tabularnewline
\hline 
 & \texttt{BUFFER} &  & La transacción actual se coloca como última transacción a mover en
la CEC. Detiene a la misma sin que pase el tiempo. &  & \tabularnewline
\hline 
 & \texttt{PREEMPT} & \texttt{a} & La transacción actual arrebata la \emph{facility} $a$ que ha sido
tomada por un \texttt{SEIZE}, sin importar las prioridades. Si fue
tomada por otro \texttt{PREEMPT}, \textbf{no }la arrebata. Interrumpe
el \texttt{ADVANCE} del\emph{ owner} de la \emph{facility.} & $a=$ id de \emph{facility }a intentar arrebatar & \tabularnewline
\hline 
 & \texttt{PREEMPT} & \texttt{a,PR{[},c,d{[},RE{]}{]}} & El arrebato se produce sólo si el \emph{owner} de la \emph{facility
$a$ }tiene menos prioridad que la transacción actual. Si se especifica
\texttt{RE}, el arrebato es definitivo. & $a=$ id de \emph{facility }a intentar arrebatar

$c=$ rótulo al que bifurca el \emph{owner} si estaba ejecutando un
\texttt{ADVANCE}

$d=$ id del parámetro del \emph{owner} donde se almacena el tiempo
remanente del \emph{owner }para terminar el \texttt{ADVANCE} & \tabularnewline
\hline 
 & \texttt{RETURN} & \texttt{a} & Retorno de recurso $a$ al \emph{owner }de dicha \emph{facility} &  & \tabularnewline
\hline 
 & \texttt{FUNAVAIL} & \multirow{1}{3cm}{\texttt{a,{[}b{]},{[}c{]},{[}d{]},{[}e{]},f,{[}g{]},h}} & Rotura de \emph{facility} $a$. El owner de la \emph{facility} bifurca
al rótulo $c$. Las transacciones que fueron interrumpidas bifurcan
al rótulo $f$. Las transacciones que están en la \emph{delay chain}
o \emph{pending chain} de $a$ bifurcan al rótulo $h$. & \begin{itemize}
\item $a=$ id de \emph{facility}
\item $b=RE$ (el \emph{owner }de la \emph{facility} la deja definitivamente)
o $CO$ (el owner continúa en poder de la \emph{facility})
\item $c$ es obligatorio si $b=RE$
\item $d=$ parámetro del \emph{owner} de la \emph{facility} donde se guarda
el tiempo remanente del \texttt{ADVANCE} que estuviera ejecutando.
\item $e=RE$ (las transacciones interrumpidas la dejan definitivamente)
o $CO$ (el owner continúa en poder de la \emph{facility})
\item $f$ es obligatorio si $e=RE$
\item $g=RE$ (las transacciones en la \emph{delay chain} o \emph{pending
chain} abandonan la idea de tomar $a$) o $CO$ (abandonan la \emph{delay
chain} o \emph{pending chain})
\item si $h$ es usado, $g$ debe ser $RE$\end{itemize}
 & \tabularnewline
\hline 
 & \texttt{FAVAIL} & \texttt{a} & Arreglo de \emph{facility} $a$ &  & \tabularnewline
\hline 
 & \texttt{SUNAVAIL} & \texttt{a} & Rotura de \emph{storage $a$} &  & \tabularnewline
\hline 
 & \texttt{SAVAIL} & \texttt{a} & Arreglo de \emph{storage $a$} &  & \tabularnewline
\hline 
 & \texttt{MATCH} & \texttt{a} & Sincronizar el movimiento de transacciones de la misma familia. & $a=$ rótulo para ser testeado por transacciones iguales. & \tabularnewline
\hline 
 & \texttt{JOIN} & \texttt{g,n} & Unir el número $n$ al grupo numérico $g$ & \begin{itemize}
\item $g$= id del grupo \textbf{numérico} o SNA
\item $n=$constante o SNA\end{itemize}
 & \tabularnewline
\hline 
 & \texttt{REMOVE} & \texttt{g,,n{[},,,f{]}} & Remover al número $n$ del grupo numérico $g$. Si no pertenece, bifurca
a $f$ & \begin{itemize}
\item $g$= id del grupo \textbf{numérico} o SNA
\item $n=$constante o SNA\end{itemize}
 & \tabularnewline
\hline 
 & \texttt{EXAMINE} & \texttt{g,n,f} & Si el número $n$ pertenece al grupo numérico $g$, continúa en el
bloque siguiente. Si no, bifurca a $f$ & \begin{itemize}
\item $g$= id del grupo \textbf{numérico} o SNA
\item $n=$constante o SNA\end{itemize}
 & \tabularnewline
\hline 
\end{longtable}

\pagebreak{}Para referirse a parámetros:
\begin{enumerate}
\item Por nombre: 

\begin{enumerate}
\item \texttt{P\$CAJAS}
\item \texttt{{*}\$CAJAS}
\item \texttt{{*}CAJAS}
\end{enumerate}
\item Por número:

\begin{enumerate}
\item \texttt{P4}
\item \texttt{{*}4}
\end{enumerate}
\end{enumerate}
Para referirse a un \texttt{SAVEVALUE}:
\begin{enumerate}
\item Por nombre: \texttt{X\$CAJAS}
\item Por número: \texttt{X4}
\end{enumerate}
Para referirse a una \texttt{VARIABLE}: 
\begin{enumerate}
\item Direccionamiento directo: \texttt{V\$CAJA, V\$1}
\item Direccionamiento indirecto: \texttt{V{*}CAJA, V{*}1}
\end{enumerate}
Para referirse a una \texttt{FUNCTION}: \texttt{FN\$FUNCION}

Para referirse a una \texttt{MATRIX}:
\begin{enumerate}
\item Por nombre: \texttt{MX\$MATRIZ(col,fil)}
\item Por número: \texttt{Mx4(col,fil)}\end{enumerate}

% Incluir los nombres de las personas que han colaborado en la creación del apunte
\colaborador{María Ines Parnisari}
\colaborador{Ezequiel Pérez Dittler}
%\revisor{Dr. Profesor}{10/01/2015}
\makeseccioncolaboradores % Crea la sección de colaboradores

% Incluir el historial de cambios
\revision{27/08/2016}{Versión inicial con el contenido del apunte de María Ines.}
\makehistorial

\end{document}
