\documentclass{article}
\usepackage[utf8]{inputenc}
\usepackage[spanish]{babel}
\usepackage[landscape,a4paper]{geometry}
\geometry{
  verbose,
  tmargin=1cm,
  bmargin=1cm,
  lmargin=1cm,
  rmargin=1cm,
  headheight=0cm,
  headsep=0cm,
  footskip=0cm
}
%\pagestyle{empty}
%\setcounter{tocdepth}{2}
%\setlength{\parskip}{\smallskipamount}
%\setlength{\parindent}{0pt}
%\usepackage{color}

\addto\shorthandsspanish{\spanishdeactivate{~<>}}

\usepackage{array}
\usepackage{longtable}
\usepackage{float}
\usepackage{multirow}
%\usepackage[unicode=true,
% bookmarks=false,
% breaklinks=false,pdfborder={0 0 1},backref=false,colorlinks=false]
% {hyperref}
%\hypersetup{pdftitle={Simulación},
% pdfauthor={María Inés Parnisari}}

\makeatletter

%%%%%%%%%%%%%%%%%%%%%%%%%%%%%% LyX specific LaTeX commands.
%% Because html converters don't know tabularnewline
\providecommand{\tabularnewline}{\\}

%%%%%%%%%%%%%%%%%%%%%%%%%%%%%% User specified LaTeX commands.
\usepackage{longtable}

\makeatother

\begin{document}

\section*{Cartilla de bloques}

\begin{longtable}{|lc>{\raggedright}p{0.13\textwidth}|>{\raggedright}p{0.2\textwidth}|>{\raggedright}p{0.2\textwidth}|>{\raggedright}p{0.2\textwidth}|}
\hline
\multicolumn{3}{|l|}{\textbf{Bloque}} & \textbf{Acción que representa} & \textbf{Parámetros} & \textbf{Valor por defecto}\tabularnewline
\hline
\endhead

\hline
& \texttt{START} & \texttt{n} &
Simular en el modelo hasta contabilizar $n$ transacciones terminadas &
$n=$ cantidad de transacciones & \tabularnewline

\hline
& \texttt{GENERATE} & \texttt{[m][,d][,i][,c][,p]} &
Arribo de transacciones independientes entre sí cada cierto tiempo $m \pm d$ o $m \times d$ &
\begin{itemize}
  \item $m=$ valor medio entre arribos (constante o SNA)
  \item Si $d \in N$, es el desvío máximo permitido.
        Si $d$ es una \texttt{FUNCTION}, se multiplica $m \times d$ para calcular el próximo nacimiento
  \item $i=$ instante en que se genera la primera transacción
  \item $c=$ cantidad máxima de transacciones a generar
  \item $p=$ prioridad de la transacción a generar
\end{itemize} &
\begin{itemize}
  \item $m=0$
  \item $d=0$
  \item $i=$ $A \pm B$ o $A \times B$
  \item $c=\infty$
  \item $p=0$ (prioridad mínima)
\end{itemize} \tabularnewline

\hline
& \texttt{ADVANCE} & \texttt{m[,d]} &
Realización de una tarea que dura cierto tiempo $m \pm d$ o $m \times d$ &
\begin{itemize}
  \item $m=$ valor medio de la duración de la tarea (constante o SNA)
  \item Si $d \in N$, es el desvío máximo permitido.
        Si $d$ es una \texttt{FUNCTION}, se multiplica $m \times d$ para calcular la duración.
        \emph{No puede ser referencia a FUNCTION}
  \item Debe cumplirse que $m \geq d$
\end{itemize} &
\begin{itemize}
  \item $m=0$
  \item $d=0$
\end{itemize} \tabularnewline

\hline
& \texttt{TERMINATE} & \texttt{[n]} &
Salida de $n$ transacciones. Si $n$ no se especifica, no se contabiliza la transacción pero la misma se destruye &
$n \in N$ & $n=0$ \tabularnewline

\hline
\pagebreak

\hline
& \texttt{SEIZE} & \texttt{f} &
Tomar un recurso de uso exclusivo (\emph{facility}) &
$f=$ identificador de la \emph{facility} (número o nombre) & \tabularnewline

\hline
& \texttt{RELEASE} & \texttt{f} &
Dejar un recurso de uso exclusivo (\emph{facility}) &
$f=$ identificador de la \emph{facility} (número o nombre) & \tabularnewline

\hline
& \texttt{PRIORITY} & \texttt{p} &
Cambia la prioridad de la transacción activa & $p=$ nueva prioridad & \tabularnewline

\hline
\texttt{NSTORAGE} & \texttt{STORAGE} & \texttt{n} &
Definir la capacidad de un \emph{storage} en $n$ transacciones &
\begin{itemize}
  \item $NSTORAGE=$ identificador del \emph{storage} (número o nombre)
  \item $n \in N$, $n < 30000$
\end{itemize} & \tabularnewline

\hline
& \texttt{ENTER} & \texttt{s[,l]} &
Intentar ocupar uno o más lugares de un recurso de uso compartido (\emph{storage}) &
\begin{itemize}
  \item $s=$ identificador del \emph{storage} (número o nombre)
  \item $l=$ lugares que ocupa la transacción
\end{itemize} &
$l=1$ \tabularnewline

\hline
& \texttt{LEAVE} & \texttt{s[,l]} &
Dejar uno o más lugares de un recurso de uso compartido (\emph{storage}).
No es necesario haber ejecutado \texttt{ENTER} previamente. &
\begin{itemize}
  \item $s=$ identificador del \emph{storage} (número o nombre)
  \item $l=$ lugares que deja la transacción
\end{itemize} &
$l=1$ \tabularnewline

\hline
& \texttt{QUEUE} & \texttt{f[,l]} &
Hacer cola para la \emph{facility} $f$ &
\begin{itemize}
  \item $f=$ identificador de la \emph{queue} (número o nombre)
  \item $l=$ lugares que ocupa la transacción
\end{itemize} &
$l=1$ \tabularnewline

\hline
& \texttt{DEPART} & \texttt{f[,l]} &
Salir de la cola de la \emph{facility} $f$.
Si la \texttt{QTABLE} respectiva está definida, registra allí las estadísticas &
\begin{itemize}
  \item $f=$ identificador de la \emph{queue} (número o nombre)
  \item $l=$ lugares que deja la transacción
\end{itemize} &
$l=1$ \tabularnewline

\hline
& \texttt{TRANSFER} & \texttt{[p,[a]],b} &
Bifurcar estocásticamente. Con probabilidad $p$ se bifurca a $b$,
y con probabilidad \textbf{$1-p$} se bifurca a $a$.
Si $a$ no se especifica, bifurca al bloque siguiente con probabilidad $1-p$.
Si ni $p$ ni $a$ se especifican, bifurca siempre a $b$ &
\begin{itemize}
  \item $0\leq p<1$, o $p$ es una referencia a una función de $n$ puntos tal que $0 \leq y_{i} < 1000$ $(i=1..n)$
  \item $a$ y \textbf{$b$} son rótulos o referencias a funciones
\end{itemize} &
$p=1$ \tabularnewline

\hline
& \texttt{TRANSFER} & \texttt{SIM,b,c} &
Si el \emph{delay indicator} de la transacción está en ON (set) \textemdash si
alguno de los \texttt{GATE/TEST} anteriores no se cumple\textemdash ,
la transacción es enviada al rótulo $c$ y pone el \emph{delay indicator}
en OFF (reset). Si el \emph{delay indicator} está en OFF \textemdash todos
los \texttt{GATE/TEST} anteriores se cumplen \textbf{simultáneamente}\textemdash ,
la transacción es enviada al rótulo $b$. &
$b$ y $c$ son rótulos. & \tabularnewline

\hline
\texttt{NTABLE} & \texttt{TABLE} & \texttt{v,x$_{0}$,$\Delta$x,n} &
Definir una tabla de distribución de frecuencia. Necesita de \texttt{TABULATE}. &
\begin{itemize}
  \item $v=$ valor a tabular de la transacción que ejecuta el \emph{tabulate} (cualquier SNA)
  \item $x_{0}=$ Límite superior del primer intervalo
  \item $\Delta x=$ tamaño de cada intervalo
  \item $n=$cantidad de intervalos
\end{itemize} & \tabularnewline

\hline
& \texttt{TABULATE} & \texttt{t} &
Agrega un valor a la \texttt{TABLE} \texttt{t} &
$t=$ identificador de la \texttt{TABLE} & \tabularnewline

\hline
\texttt{NQTABLE} & \texttt{QTABLE} & \texttt{ntable,t$_{0}$,$\Delta t$,n} &
Define una tabla de distribución de frecuencias para tiempos en cola. No necesita de \texttt{TABULATE}. &
\begin{itemize}
  \item $NQTABLE=$ identificador de la \emph{queue} (número o nombre)
  \item $t_{0}=$ límite superior del primer intervalo
  \item $\Delta t=$ tamaño de cada intervalo
  \item $n=$ cantidad de intervalos
\end{itemize} & \tabularnewline

\hline
& \texttt{SAVEVALUE} &
\texttt{s,v}

\texttt{s+,v}

\texttt{s-,v} &
Asigna, suma o resta el valor $v$ al \emph{savevalue} $s$ &
\begin{itemize}
  \item $s=$ identificador del \emph{savevalue} (número o nombre)
  \item $v=$ valor a asignar, sumar o restar
\end{itemize} & \tabularnewline

\hline
& \texttt{INITIAL} & \texttt{s,a} &
Asigna el valor inicial $a$ al \emph{savevalue} $s$ &
\begin{itemize}
  \item $s=$ identificador del \emph{savevalue} (número o nombre)
  \item $v=$ valor con el que se inicializa
\end{itemize} & \tabularnewline

\hline
& \texttt{ASSIGN} &
\texttt{p,v}

\texttt{p+,v}

\texttt{p-,v} &
Asigna, suma o resta el valor $v$ al parámetro $p$ de la transacción activa &

\begin{itemize}
  \item $p=$ identificador del parámetro (puede ser un SNA)
  \item $v=$ valor a asignar, sumar o restar (puede ser una constante o un SNA)
\end{itemize} & \tabularnewline

\hline
& \texttt{MARK} & \texttt{[p]} &
Se almacena un M1 ``paralelo'' en el parámetro $p$ de la transacción.
Si $p$ no se especifica, se reinicializa M1 en cero. &
$p=$ id del parámetro & $p=M1$ \tabularnewline

\hline
\texttt{SELECT} & \texttt{op logico}\footnote{
  \texttt{Logic Switch: LS, LR}

  \texttt{Facility: U, NU, I, NI, FV, FNV}

  \texttt{Storage: SE, SNE, SF, SNF, SV, SNV}
} & \texttt{p,v$_{1}$,v$_{2}$[,,,f]} &
Busca una \emph{facility}, \emph{storage} o \emph{logic switch} según el
\texttt{op logico} y la almacena en el parámetro $p$ de la transacción activa.
Al finalizar, $v_{1}\leq a\leq v_{2}$ o, si no pudo seleccionar, bifurca a $f$.
Si $f$ no esta, establece que $p=0$ y continúa en el bloque siguiente &
\begin{itemize}
  \item $p=$ identificador del parámetro
  \item $v_{1}=$ número de objeto menor desde el cual se hace el \texttt{SELECT}
  \item $v_{2}=$ número de objeto mayor desde el cual se hace el \texttt{SELECT}
  \item $f=$ rótulo al que bifurca si la selección fue infructuosa
\end{itemize} & \tabularnewline

\hline
\texttt{SELECT} & \texttt{op relacion}\footnote{\texttt{EQ, L, LE, G, GE, NE}} &
\texttt{a,v$_{1}$,v$_{2}$,d,e{[},f{]}} &
Busca una objeto $e$ según el \texttt{op relacion} y lo almacena en
el parámetro $a$ de la transacción activa. Al finalizar, $v_{1} \leq a \leq v_{2}$
o, si no hay desocupadas, bifurca a $f$. Si $f$ no esta, establece que $a=0$ y
continúa en el bloque siguiente &
\begin{itemize}
  \item $a=$ identificador del parámetro
  \item $v_{1}=$ número de objeto menor desde el cual se hace el \texttt{SELECT}
  \item $v_{2}=$ número de objeto mayor desde el cual se hace el \texttt{SELECT}
  \item $d=$ valor con el que se compara
  \item $e=$ clase de objeto que se compara (SNAs)
  \item $f=$ rótulo al que bifurca si la selección fue infructuosa
\end{itemize} & \tabularnewline

\hline
\texttt{SELECT} & \texttt{MIN/MAX} & \texttt{p,v$_{1}$,v$_{2}$,,e} &
Busca un \emph{facility} desocupado según el criterio $e$ y lo almacena en $p$.
Si el criterio no alcanza para definir, $p=v_{1}$ &
\begin{itemize}
  \item $p=$ identificador del parámetro
  \item $v_{1}=$ número menor del objeto a consultar
  \item $v_{2}=$ número mayor del objeto a consultar
  \item $e=$ clase de objeto en la que se busca el mínimo o máximo
\end{itemize} & \tabularnewline

\hline
\texttt{NVARIABLE} & \texttt{VARIABLE} & \texttt{f} & Define una función que se evalúa cada vez que una transacción hace
referencia a ella. Se truncan los decimales de los resultados intermedios
y del resultado final. & $f=$ función con operadores (cualquier SNA)

\noindent \centering{}%
\begin{tabular}{|c|c|}
\hline 
+ & suma\tabularnewline
\hline 
\hline 
- & resta\tabularnewline
\hline 
\hline 
\# & multiplicación\tabularnewline
\hline 
\hline 
/ & división\tabularnewline
\hline 
\hline 
@ & módulo\tabularnewline
\hline 
\hline 
\textasciicircum{} & potencia\tabularnewline
\hline 
\end{tabular} & \tabularnewline
\hline 
\texttt{NFUNCTION} & \texttt{FUNCTION} & \texttt{x,tn}

\texttt{x$_{1}$,y$_{1}$/.../x$_{n}$,y$_{n}$} & Devuelve un valor cada vez que se la invoca. Define una función de
$n$ puntos. Si $x\leq x_{1}$, devuelve $y_{1}$, si $x\geq x_{n}$,
devuelve $y_{n}$. Se truncan los decimales de los resultados intermedios
y del resultado final. & \begin{itemize}
\item $x=$ variable independiente (cualquier SNA). Si $x$ es RN, $0<x_{i}\leq1$
\item $t=$ tipo de la función ($C=$continua, $D=$discreta, $E=$discreta
de atributos numéricos)
\footnote{Debe cumplirse que $x_{1}\leq x_{2}\leq...\leq x_{n}$.
Si es \textbf{continua}, y $x_{i}<x<x_{i+1}$ se interpola linealmente
entre $y_{i}$ e $y_{i+1}$. Si es \textbf{discreta}, y $x_{i}<x<x_{i+1}$,
se devuelve $y_{i+1}$ (los $y$ pueden ser rótulos). Si es \textbf{discreta
de atributos numéricos}, es como la discreta pero las $y$ deben ser
SNAs.}
\item $n=$ cantidad de puntos que definen la función\end{itemize}
 & \tabularnewline
\hline 
 & \texttt{LOOP} & \texttt{p,a} & Resta 1 al parámetro $p$. Si $p>0$, bifurca al rótulo $a$. Si $p=0$,
continúa en el bloque siguiente & \begin{itemize}
\item $p=$ id del parámetro a decrementar
\item $a=$ rótulo al que bifurca\end{itemize}
 & \tabularnewline
\hline 
\texttt{NMATRIX} & \texttt{MATRIX} & \texttt{,f,c} & Define una matriz & \begin{itemize}
\item $f=$ cantidad de filas
\item $c=$ cantidad de columnas\end{itemize}
 & \tabularnewline
\hline 
 & \texttt{MSAVEVALUE} & \texttt{m,f,c,v}

\texttt{m+,f,c,v}

\texttt{m-,f,c,v} & Asigna, suma o resta el valor $v$ del elemento $m[f,c]$ & \begin{itemize}
\item $m=$ id de la matriz
\item $f=$ fila (cualquier SNA)
\item $c=$ columna (cualquier SNA)
\item $v=$ valor a asignar, sumar o restar\end{itemize}
 & \tabularnewline
\hline 
\texttt{TEST} & \texttt{op} & \texttt{a,b{[},f{]}} & Si $a\,op\,b$ es V, continúa en el bloque siguiente. Si es F, bifurca
al rótulo $f$; si es F y no se especifico $f$, bloquea la transacción
hasta que sea V & \begin{itemize}
\item $op$ puede ser $e$, $ne$, $g$, $ge$, $l$, $le$
\item $a$ y $b$ pueden ser constantes o SNAs
\item $f=$ rótulo al que bifurca\end{itemize}
 & \tabularnewline
\hline 
 & \texttt{SPLIT} & \texttt{c,r{[},n{]}} & Genera $c$ nuevas transacciones, \textquotedblleft clonando\textquotedblright{}
la transacción activa (M1, PR, parámetros). Las copias bifurcan a
$r$ y el original continúa en el bloque siguiente & \begin{itemize}
\item $c=$ cantidad de copias (constante o SNA)
\item $r=$ rótulo al que bifurcan las copias
\item $n=$ id de parámetro a utilizar para enumerar el original y las copias\end{itemize}
 & \tabularnewline
\hline 
 & \texttt{ASSEMBLE} & \texttt{a} & Reúne un grupo de transacciones y deja pasar solo una transacción
que representa al grupo, las restantes $a-1$ son destruidas & $a=$ cantidad de miembros de la familia a sincronizar & \tabularnewline
\hline 
 & \texttt{GATHER} & \texttt{m} & Reúne un grupo de transacciones y deja pasar a todas una vez reunida
la cantidad especificada, sin destruir ninguna transacción & $m=$ cantidad de miembros de la familia a reunir & \tabularnewline
\hline 
\texttt{LOGIC} & \texttt{v} & \texttt{ll} & Cambia el estado del \emph{logic switch} $ll$ por el valor de $v$ & \begin{itemize}
\item $v=$ S (set), R (reset), I (invertido)
\item $ll=$ id de la llave\end{itemize}
 & \tabularnewline
\hline 
 & \texttt{INITIAL} & \texttt{rll{[},v{]}} & Inicializa el estado de la llave $rll$ con el valor $v$. & \begin{itemize}
\item $rll=$\textbf{ referencia} a una llave lógica
\item $v=$ 1 (set), 0 (reset)\end{itemize}
 & $v=1$(set)\tabularnewline
\hline 
\texttt{GATE} & \texttt{op lógico\ref{op_logicos}} & \texttt{a{[},f{]}} & Consulta el estado del \emph{logic switch}, \emph{facility}, \emph{storage}
o rótulo $a$. Si $op\,A$ es V, continúa en el bloque siguiente.
Si es F, bifurca a $f$. Si no se especifica $f$, la transacción
se bloquea hasta que sea V (\emph{delay indicator }en ON) & \begin{itemize}
\item $a=$ id del \emph{logic switch}, \emph{facility}, \emph{storage}
\item $f=$ rótulo al que bifurca si la condición es falsa.\end{itemize}
 & \tabularnewline
\hline 
\texttt{COUNT} & \texttt{op logico}\footnote{\label{op_logicos}Logic Switch: LS, LR

Facility: U, NU, I, NI, FV, FNV

Storage: SE, SNE, SF, SNF, SV, SNV} & \texttt{a,v$_{1}$,v$_{2}$} & Cuenta la cantidad de objetos desde $v_{1}$ hasta $v_{2}$ que cumplen
la condición $op\,objeto$, y lo almacena en $a$. & \begin{itemize}
\item $a=$ parámetro que recibe la cantidad de objetos que cumplen la condición
\item $v_{1}=$ número menor del objeto a consultar
\item $v_{2}=$ número mayor del objeto a consultar\end{itemize}
 & \tabularnewline
\hline 
\texttt{COUNT} & \texttt{op relacion}\footnote{\texttt{EQ, L, LE, G, GE, NE}} & \texttt{a,v$_{1}$,v$_{2}$,d,e} & Cuenta la cantidad de objetos desde $v_{1}$ hasta $v_{2}$ que cumplen
la condición $objeto\,op\,d$, y lo almacena en $a$. & \begin{itemize}
\item $a=$ parámetro que recibe la cantidad de objetos que cumple la condición
\item $v_{1}=$ número menor del objeto a consultar
\item $v_{2}=$ número mayor del objeto a consultar
\item $d=$ valor con el que se compara
\item $e=$ clase de objeto que se selecciona (SNA)\end{itemize}
 & \tabularnewline
\hline 
\texttt{COUNT} & \texttt{MIN/MAX} & \texttt{a,v$_{1}$,v$_{2}$,,e} & Cuenta la cantidad de objetos desde $v_{1}$ hasta $v_{2}$ que son
mínimo o máximo, y lo almacena en $a$. & \begin{itemize}
\item $a=$ parámetro que recibe la cantidad de objetos que son mínimo o
máximo
\item $v_{1}=$ número menor del objeto a consultar
\item $v_{2}=$ número mayor del objeto a consultar
\item $e=$ clase de objeto que se compara (SNA)\end{itemize}
 & \tabularnewline
\hline 
 & \texttt{JOIN} & \texttt{g} & La transacción activa se une al grupo $g$ ``personalmente''. & $g=$ id del grupo. & \tabularnewline
\hline 
 & \texttt{EXAMINE} & \texttt{g,,f} & Verifica si la transacción activa pertenece al grupo $g$. Si pertenece,
continúa en el bloque siguiente. Si no, bifurca a $f$. & \begin{itemize}
\item $g=$ id del grupo.
\item $f=$ rótulo al que bifurca.\end{itemize}
 & \tabularnewline
\hline 
 & \texttt{REMOVE} & \texttt{g} & La transacción activa se elimina al grupo $g$ ``personalmente''. & $g=$ id del grupo. & \tabularnewline
\hline 
\texttt{REMOVE} & \texttt{{[}op{]}}\footnote{\texttt{E,G,GE,L,LE,MAX,MIN,NE}} & \texttt{g,{[}c{]},,{[}p{]},{[}v{]},{[}f{]}} & Intenta eliminar $c$ transacciones del grupo $g$ que cumplen la
condición $p\,op\,v$. Si no puede eliminar todas, bifurca a $f$.  & \begin{itemize}
\item $g=$ id del grupo
\item $c=$cantidad de transacciones a eliminar
\item $p=$ idel parámetro que se consulta de la transacción
\item $v=$valor con el que se compara (no se pone si $op=MIN/MAX$)
\item $f=$rótulo al que bifurca si no pudo eliminar $c$ transacciones\end{itemize}
 & $op=E$

$c=ALL$\tabularnewline
\hline 
\texttt{ALTER} & \texttt{{[}op{]}}\footnote{\texttt{E,G,GE,L,LE,NE}} & \multirow{1}{3cm}{\texttt{g,c,p$_{1}$,v$_{1}$,{[}p$_{2}${]},{[}v$_{2}${]},{[}f{]}}} & Intenta alterar $c$ transacciones del grupo $g$ de la siguiente
forma: para cada transacción, si $p_{2}\,op\,v_{2}$ es V, se asigna
$p_{1}=v_{1}$. Si no puede alterar $c$ transacciones, bifurca al
rótulo $f$. & \begin{itemize}
\item $g=$ id del grupo
\item $c=$ cantidad de transacciones a eliminar\footnote{$c$ puede ser \texttt{ALL}}
\item $p_{1}=$ id del parámetro a alterar
\item $v_{1}=$valor a asignar al parámetro $p_{1}$ (constante o SNA)
\item $p_{2}=$ id del parámetro a consultar
\item $v_{2}=$ valor con el que se compara $p_{2}$ (constante o SNA)
\item $f=$rótulo al que bifurca si no pudo alterar $c$ transacciones\end{itemize}
 & $op=E$\tabularnewline
\hline 
\texttt{SCAN} & \texttt{{[}op{]}} & \texttt{g,b,{[}c{]},p$_{1}$,p$_{2}${[},f{]}} & Busca en el grupo $g$ la \textbf{primera} transacción que cumple
la condición $b\,op\,c$. Copia el parámetro $p_{1}$ en el parámetro
$p_{2}$ de la transacción actual. Si no encuentra ninguna, bifurca
a $f$. & \begin{itemize}
\item $OP$ puede ser $e,g,ge,l,le,max,min,ne$
\item $b=$ id del parámetro que se consulta
\item $c=$ valor con el que se compara (si $OP$ es $min/max$, no se pone)
\item $p_{1}=$ id del parámetro de la transacción del grupo que se copia
\item $p_{2}=$ id del parámetro de la transacción actual que recibe el
valor de $p_{1}$\end{itemize}
 & \begin{itemize}
\item $op=E$\end{itemize}
\tabularnewline
\hline 
 & \texttt{LINK} & \texttt{c,r{[},f{]}} & Encadena la transacción pasiva actual a la cadena $c$ según la regla
$r$. Si se especifica el rótulo $f$, la primera transacción \textbf{no
es encadenada} y bifurca al rótulo $f$. & \begin{itemize}
\item $c=$ id de la cadena (constante o SNA)
\item $r=$ regla por la cual se encadena (FIFO, LIFO, PR, Pn, M1)
\item $f=$ rótulo al que va la primera transacción\end{itemize}
 & \tabularnewline
\hline 
\texttt{UNLINK} & \texttt{{[}op{]}} & \texttt{a,b,c,{[}d{]},{[}e{]},{[}f{]}} & La transacción activa actual desencadena $c$ transacciones de la
cadena $a$ tales que se verifique $d\,op\,e$, y las envía al rótulo
$b$. Si no puede liberar todas, bifurca a $f$ (las que fueron liberadas,
quedan liberadas). & \begin{itemize}
\item $op$ puede ser \texttt{E,G,GE,L,LE,NE}
\item $a=$ id de la cadena
\item $b=$ rótulo al que se envían las transacciones liberadas
\item $c=$ cantidad de transacciones que se desencadenan\footnote{$c$ puede ser \texttt{ALL}}
\item $d=$ id de parámetro de la transacción\textbf{ activa} que se consulta
\item $e=$\textbf{ referencia} del parámetro de la transacción \textbf{pasiva}
que se consulta
\item $f=$ rótulo al que bifurca la transacción activa si no pudo liberar
$c$ transacciones.\end{itemize}
 & \begin{itemize}
\item $op=E$\end{itemize}
\tabularnewline
\hline 
 & \texttt{BUFFER} &  & La transacción actual se coloca como última transacción a mover en
la CEC. Detiene a la misma sin que pase el tiempo. &  & \tabularnewline
\hline 
 & \texttt{PREEMPT} & \texttt{a} & La transacción actual arrebata la \emph{facility} $a$ que ha sido
tomada por un \texttt{SEIZE}, sin importar las prioridades. Si fue
tomada por otro \texttt{PREEMPT}, \textbf{no }la arrebata. Interrumpe
el \texttt{ADVANCE} del\emph{ owner} de la \emph{facility.} & $a=$ id de \emph{facility }a intentar arrebatar & \tabularnewline
\hline 
 & \texttt{PREEMPT} & \texttt{a,PR{[},c,d{[},RE{]}{]}} & El arrebato se produce sólo si el \emph{owner} de la \emph{facility
$a$ }tiene menos prioridad que la transacción actual. Si se especifica
\texttt{RE}, el arrebato es definitivo. & $a=$ id de \emph{facility }a intentar arrebatar

$c=$ rótulo al que bifurca el \emph{owner} si estaba ejecutando un
\texttt{ADVANCE}

$d=$ id del parámetro del \emph{owner} donde se almacena el tiempo
remanente del \emph{owner }para terminar el \texttt{ADVANCE} & \tabularnewline
\hline 
 & \texttt{RETURN} & \texttt{a} & Retorno de recurso $a$ al \emph{owner }de dicha \emph{facility} &  & \tabularnewline
\hline 
 & \texttt{FUNAVAIL} & \multirow{1}{3cm}{\texttt{a,{[}b{]},{[}c{]},{[}d{]},{[}e{]},f,{[}g{]},h}} & Rotura de \emph{facility} $a$. El owner de la \emph{facility} bifurca
al rótulo $c$. Las transacciones que fueron interrumpidas bifurcan
al rótulo $f$. Las transacciones que están en la \emph{delay chain}
o \emph{pending chain} de $a$ bifurcan al rótulo $h$. & \begin{itemize}
\item $a=$ id de \emph{facility}
\item $b=RE$ (el \emph{owner }de la \emph{facility} la deja definitivamente)
o $CO$ (el owner continúa en poder de la \emph{facility})
\item $c$ es obligatorio si $b=RE$
\item $d=$ parámetro del \emph{owner} de la \emph{facility} donde se guarda
el tiempo remanente del \texttt{ADVANCE} que estuviera ejecutando.
\item $e=RE$ (las transacciones interrumpidas la dejan definitivamente)
o $CO$ (el owner continúa en poder de la \emph{facility})
\item $f$ es obligatorio si $e=RE$
\item $g=RE$ (las transacciones en la \emph{delay chain} o \emph{pending
chain} abandonan la idea de tomar $a$) o $CO$ (abandonan la \emph{delay
chain} o \emph{pending chain})
\item si $h$ es usado, $g$ debe ser $RE$\end{itemize}
 & \tabularnewline
\hline 
 & \texttt{FAVAIL} & \texttt{a} & Arreglo de \emph{facility} $a$ &  & \tabularnewline
\hline 
 & \texttt{SUNAVAIL} & \texttt{a} & Rotura de \emph{storage $a$} &  & \tabularnewline
\hline 
 & \texttt{SAVAIL} & \texttt{a} & Arreglo de \emph{storage $a$} &  & \tabularnewline
\hline 
 & \texttt{MATCH} & \texttt{a} & Sincronizar el movimiento de transacciones de la misma familia. & $a=$ rótulo para ser testeado por transacciones iguales. & \tabularnewline
\hline 
 & \texttt{JOIN} & \texttt{g,n} & Unir el número $n$ al grupo numérico $g$ & \begin{itemize}
\item $g$= id del grupo \textbf{numérico} o SNA
\item $n=$constante o SNA\end{itemize}
 & \tabularnewline
\hline 
 & \texttt{REMOVE} & \texttt{g,,n{[},,,f{]}} & Remover al número $n$ del grupo numérico $g$. Si no pertenece, bifurca
a $f$ & \begin{itemize}
\item $g$= id del grupo \textbf{numérico} o SNA
\item $n=$constante o SNA\end{itemize}
 & \tabularnewline
\hline 
 & \texttt{EXAMINE} & \texttt{g,n,f} & Si el número $n$ pertenece al grupo numérico $g$, continúa en el
bloque siguiente. Si no, bifurca a $f$ & \begin{itemize}
\item $g$= id del grupo \textbf{numérico} o SNA
\item $n=$constante o SNA\end{itemize}
 & \tabularnewline
\hline 
\end{longtable}

\pagebreak{}Para referirse a parámetros:
\begin{enumerate}
\item Por nombre: 

\begin{enumerate}
\item \texttt{P\$CAJAS}
\item \texttt{{*}\$CAJAS}
\item \texttt{{*}CAJAS}
\end{enumerate}
\item Por número:

\begin{enumerate}
\item \texttt{P4}
\item \texttt{{*}4}
\end{enumerate}
\end{enumerate}
Para referirse a un \texttt{SAVEVALUE}:
\begin{enumerate}
\item Por nombre: \texttt{X\$CAJAS}
\item Por número: \texttt{X4}
\end{enumerate}
Para referirse a una \texttt{VARIABLE}: 
\begin{enumerate}
\item Direccionamiento directo: \texttt{V\$CAJA, V\$1}
\item Direccionamiento indirecto: \texttt{V{*}CAJA, V{*}1}
\end{enumerate}
Para referirse a una \texttt{FUNCTION}: \texttt{FN\$FUNCION}

Para referirse a una \texttt{MATRIX}:
\begin{enumerate}
\item Por nombre: \texttt{MX\$MATRIZ(col,fil)}
\item Por número: \texttt{Mx4(col,fil)}\end{enumerate}

\end{document}
